% ***********************************************************
% ******************* PHYSICS HEADER ************************
% ***********************************************************
% Version 2
\documentclass[11pt]{article} 
\usepackage{amsmath} % AMS Math Package
\usepackage{amsthm} % Theorem Formatting
\usepackage{amssymb}	% Math symbols such as \mathbb
\usepackage{graphicx} % Allows for eps images
\usepackage{multicol} % Allows for multiple columns
\usepackage[dvips,letterpaper,margin=0.75in,bottom=0.5in]{geometry}
 % Sets margins and page size
\pagestyle{empty} % Removes page numbers
\makeatletter % Need for anything that contains an @ command 
\renewcommand{\maketitle} % Redefine maketitle to conserve space
{ \begingroup \vskip 10pt \begin{center} \large {\bf \@title}
	\vskip 10pt \large \@author \hskip 20pt \@date \end{center}
  \vskip 10pt \endgroup \setcounter{footnote}{0} }
\makeatother % End of region containing @ commands
\renewcommand{\labelenumi}{(\alph{enumi})} % Use letters for enumerate
% \DeclareMathOperator{\Sample}{Sample}
\let\vaccent=\v % rename builtin command \v{} to \vaccent{}
\renewcommand{\v}[1]{\ensuremath{\mathbf{#1}}} % for vectors
\newcommand{\gv}[1]{\ensuremath{\mbox{\boldmath$ #1 $}}} 
% for vectors of Greek letters
\newcommand{\uv}[1]{\ensuremath{\mathbf{\hat{#1}}}} % for unit vector
\newcommand{\abs}[1]{\left| #1 \right|} % for absolute value
\newcommand{\avg}[1]{\left< #1 \right>} % for average
\let\underdot=\d % rename builtin command \d{} to \underdot{}
\renewcommand{\d}[2]{\frac{d #1}{d #2}} % for derivatives
\newcommand{\dd}[2]{\frac{d^2 #1}{d #2^2}} % for double derivatives
\newcommand{\pd}[2]{\frac{\partial #1}{\partial #2}} 
% for partial derivatives
\newcommand{\pdd}[2]{\frac{\partial^2 #1}{\partial #2^2}} 
% for double partial derivatives
\newcommand{\pdc}[3]{\left( \frac{\partial #1}{\partial #2}
 \right)_{#3}} % for thermodynamic partial derivatives
\newcommand{\ket}[1]{\left| #1 \right>} % for Dirac bras
\newcommand{\bra}[1]{\left< #1 \right|} % for Dirac kets
\newcommand{\braket}[2]{\left< #1 \vphantom{#2} \right|
 \left. #2 \vphantom{#1} \right>} % for Dirac brackets
\newcommand{\matrixel}[3]{\left< #1 \vphantom{#2#3} \right|
 #2 \left| #3 \vphantom{#1#2} \right>} % for Dirac matrix elements
\newcommand{\grad}[1]{\gv{\nabla} #1} % for gradient
\let\divsymb=\div % rename builtin command \div to \divsymb
\renewcommand{\div}[1]{\gv{\nabla} \cdot #1} % for divergence
\newcommand{\curl}[1]{\gv{\nabla} \times #1} % for curl
\let\baraccent=\= % rename builtin command \= to \baraccent
\renewcommand{\=}[1]{\stackrel{#1}{=}} % for putting numbers above =
\newtheorem{prop}{Proposition}
\newtheorem{thm}{Theorem}[section]
\newtheorem{lem}[thm]{Lemma}
\theoremstyle{definition}
\newtheorem{dfn}{Definition}
\theoremstyle{remark}
\newtheorem*{rmk}{Remark}
\usepackage[utf8x]{inputenc}
\usepackage[italian]{babel}
\usepackage{graphicx}
\usepackage{latexsym}
\usepackage{verbatim} % commenti estesi
\newcommand{\p}[2]{ \frac{ \partial #1}{ \partial #2}} %derivata partiale
\newcommand{\lap}{\nabla^2} %laplaciano
\newcommand{\ham}{\mathcal{H}} % Simbolo dell'hamiltoniana
\newcommand{\rint}{\int_\mathbb{R}} % Integrale su R
\newcommand{\modq}[1]{| #1|^2} % Modulo quadro di ``argomento''
\newcommand{\navg}[2]{\left< #1 ^{#2} \right>} %media di ``arg''^n
\newcommand{\con}[1]{\overline{#1}} %coniugato con la barra
\newcommand{\ih}{\frac{i}{\hbar}} % i/h tagliato

% ***********************************************************
% ********************** END HEADER *************************
% ***********************************************************
%\documentclass[a4paper,10pt]{article}

\title{Formulario di Meccanica Quantistica}
\author{}

\begin{document}

\maketitle


\section{Parte I: matematica/princìpi}
Data una funzione d'onda $\psi(x)$, (ampiezza di probabilità) per essere una $p.d.f.$ per una osservabile deve essere normalizzata a uno,
ossia: $ \int_\mathbb{R} | \psi(x) |^2\; dx = 1 $. \\
La probabilità si ricava come $ P(x=x_0 + dx)= | \psi(x_0) |^2 $. \\
\subsection{Trasformata di Fourier in MQ}
$$
\hat{\psi}(p)= \frac{1}{\sqrt{2 \pi \hbar}^3} \int_\mathbb{R} d^3 x \, \psi(x,0) e^{-\frac{i}{\hbar} \v{p}\cdot {\v{x}}}$$ 
Proprietà:
$$
 I) \quad \hat{\d{f}{x}} = i \frac{p}{h} \hat{f}(p)  \qquad \qquad II) \quad \hat{(xf)} = i \hbar \d{}{p}\hat{f}(p)
$$
$$
III) \quad \psi(x)= \frac{1}{\sqrt{2 \pi \hbar}^3} \int_\mathbb{R} d^3 p \, \hat{\psi}(x,0) e^{\frac{i}{\hbar} \v{p}\cdot {\v{x}}}$$ 
$$
    IV) \quad \int_\mathbb{R} \overline{g(x)} f(x) dx =  \rint \overline{\hat{g}(p)}  \hat{f}(p) dp \qquad \Longrightarrow \qquad \rint |f(x)|^2 dx = \rint |\hat{f}(p)|^2 dp
$$
Medie $m-esime$ e ``spread'' data una funzione ampiezza di probabilità:
$$ \navg{x}{n} = \rint x^n \modq{\psi} \qquad \qquad \Delta x = \sqrt{ \navg{x}{2} - \avg{x}^2}
$$ 
\subsection{Basi discrete e continue}
Una insieme di vettori è sonc se:
$$ 
 \bra{n}\ket{m} = \delta_{mn} \qquad \sum_n \ket{n}\bra{n} = \mathbb{I}
$$
con $\mathbb{I} $ operatore Identità.
Per le basi continue:
$$
\rint dx \, \con{\varphi_p (x)} \varphi_p' (x)  = \delta(p-p') \qquad 
\rint dp \, \ket{\varphi_p} \bra{\varphi_p}  = \mathbb{I} 
$$

\subsection{Velocità di fase e di gruppo}
$$
v_f = \frac{\omega}{k} \qquad \qquad \qquad v_g= \d{\omega}{k} = v_{classica}
$$
\subsection{$\delta$ di Dirac}
Definizione:
$$
\rint \delta(x-x_0)f(x)dx = f(x_0) \qquad \approx  \qquad 
\delta(x) = \begin{cases}
            0,\quad x\neq0 \\
	    \infty, \quad  x= 0 
\end{cases}
$$
Caso in cui gli zeri della $\delta$ definiti dagli zeri di una generica funzione $g(x)$ con $n$ zeri ($x_i$):
$$
\rint \delta(g(x))f(x)dx = \rint \sum_{i=1}^n \frac{\delta(x-x_i)}{|g'(x_i)|} f(x) dx = \sum_{i=1}^n \frac{f(x_i)}{|g'(x_i)|}
$$
Piccola formula priva di senso (non dovrebbe converger):
$$ \rint \frac{dk}{2 \pi} e^{ikx} = \delta(x) $$
\subsection{Trucco per Trasformata di Fourier in sferiche, con $\psi(r)$}
Concettualmente: ho $\psi(x)$ e voglio ottenere $\hat{\psi}(p)$, pensandola con $p$ fissato. Vedendola così, metto p nella direzione
dell'asse $\uv{z}$, ``privilegiato'' nelle sferiche.
$$
\hat{\psi}(p)= \frac{1}{\sqrt{2 \pi \hbar}^3} \int_\mathbb{R} d^3 x \, \psi(x,0) e^{-\frac{i}{\hbar} \v{p}\cdot {\v{x}}} =
 \frac{1}{\sqrt{2 \pi \hbar}^3} \int_\mathbb{R} \psi(r) e^{-\frac{i}{\hbar} pr \cos(\theta) } r^2 \sin(\theta) d\theta d\phi =
$$  
Si pone $ t = \cos(\theta)$, e si sbatte via il $\sin(\theta)$
\section{Accenni di MQ}
\subsection{Evoluzione temporale di un pacchetto}
Data la $\hat{psi}$ ottengo la $ \psi(x,t)$:
$$
\hat{\psi} \Longrightarrow \psi(x,t) = \rint \frac{dp}{\sqrt{2 \pi \hbar}} e^{ \frac{i}{\hbar} (px - \frac{p^2 t}{2m})}
$$
\subsubsection{Pacchetto con distribuzione momenti gaussiana}
$$
\hat{\psi} =  \left( \frac{a^2}{2 \pi \hbar^2}\right)^{\frac{1}{4}} e^{- \frac{a^2}{4 \hbar^2} (p-p_0)^2}
$$
$$
\psi(x,t)= \left(\frac{2a^2}{\pi} \right)^{\frac{1}{4}} \frac{1}{ \sqrt{ a^2 + \frac{ 2 i \hbar t }{m}}} e^{- \frac{(x-\frac{p_0t}{m})}{a^2 + \frac{2i\hbar t}{m}}
\frac{i}{\hbar} p_0 x  - 	\frac{i}{\hbar} 	\frac{p_0^2 t}{2m} }
$$
\subsection{Equazione di Schr\"{o}dinger}
$$ 
  i\hbar \p{}{t}\ket{\psi} \, = -\frac{\hbar^2}{2m}\ket{\psi} + V \ket{\psi}  = \ham \ket{\psi}
$$
Inoltre si ricava che:
$$
E \geq V_{min}
$$
dove $V_{min}$ è il minimo del potenziale.
\subsubsection{Equazione di continuità}
$$
\pd{\modq{\psi}}{t} + \div{\left( \con{\psi} \, \lap\psi - \psi \, \lap\con{\psi} \right) } = 0
$$
\subsection{Problemi stazionari monodimensionali}
Caso in cui $ V= V(x)$, ossia indipendente dal tempo:
$$	\psi(x,t) \, = \, \varphi(x) \, e^{ -\frac{i}{\hbar} E t } \qquad \qquad \varphi(x) \in L^2(\mathbb{R}) 
$$	 
Sbattendo dentro l'ansatz nell'equazione di Schr\"{o}dinger otteniamo la SSE (stationary Schr\"{o}dinger equation):
$$
-\frac{\hbar}{2m} \, \lap \varphi(x) + V(x) \varphi(x) = E \varphi(x)
$$ 
\subsubsection{Condizioni di raccordo}
\begin{description}
 \item[$V(x)$ discontinuo in $x_0$ : ] $\qquad \varphi ' (x) \in \mathcal{C}^1 $
  \item[$V(x) = \infty$ in $x_0$ : ] $	\qquad \varphi(x_0) = 0 $
 \item[$V(x) = \alpha \, \delta (x-x_0)$ : ] $\qquad \varphi'(x_{0}^{+}) - \varphi(x_{0}^{-}) = -\frac{2 m \alpha}{\hbar^2} \varphi(x_0)$
\end{description}
\subsubsection{Coefficienti di trasmissione e riflessione}
$$
T = \left| \frac{J_{trasm}}{J_{inc}}\right|  \qquad \qquad R = \left| \frac{J_{rifl}}{J_{inc}} \right| \qquad \Longrightarrow \qquad T + R = 1 
$$ 


\section{Operatori}
\subsection{Momento angolare ``orbitale''}
$$
L_x = -i\hbar \left(y {\partial\over \partial z} - z {\partial\over \partial y}\right)
$$

$$
L_y = -i\hbar \left(z {\partial\over \partial x} - x {\partial\over \partial z}\right)
$$

$$ L_z = -i\hbar \left(x {\partial\over \partial y} - y {\partial\over \partial x}\right)
$$
Soddisfano il commutatore dei momenti angolari:
$$
\left[\hat{L_i},\hat{L_j}\right]= i \hbar \epsilon_{ijk} \hat{L_k}
$$


Generica matrice di rotazione:
$$
R_{xyz}(\alpha, \beta, \gamma) = \begin{pmatrix} \cos \alpha \cos \beta \cos \gamma - 
\sin \alpha \sin \gamma & \sin \alpha \cos \beta \cos \gamma + \cos \alpha \sin 
\gamma & -\sin \beta \cos \gamma \\ -\cos \alpha \cos \beta \sin \gamma & -
\sin \alpha \cos \beta \sin \gamma + \cos \alpha \cos \gamma & \sin \beta \sin \gamma \\ 
\cos \alpha \sin \beta & \sin \alpha \sin \beta & \cos \beta \end{pmatrix}
$$
Rotazione intorno all'asse z:
$$
R_z(\alpha) = \begin{pmatrix} \cos \alpha & -\sin \alpha & 0 \\ \sin \alpha & \cos \alpha & 0 \\ 0 & 0 & 1 \end{pmatrix}
$$
\section{Basi ortonormali varie}
\subsection{Polinomi di Legendre}
$$
{d \over dx} \left[ (1-x^2) {d \over dx} P(x) \right] + n(n+1)P(x) = 0
$$
La formula generale è:
$$
P_n(x) = \frac{1}{(2^n n!)}{d^n \over dx^n } \left[ (x^2-1)^n \right] 
$$
e i primi polinomi sono:
$$
	P_0(x)=1 \qquad	P_1(x)=x	 \qquad 	P_2(x) = \frac{1}{2}(3x^2 - 1)		\qquad P_3(x) = \frac{1}{2}(5x^3 - 3x)
$$
$$
	\qquad P_4(x) = \frac{1}{8}(35x^4 - 30x^2 + 3) \qquad P_5(x) = \frac{1}{8}(63x^5 - 70x^3 + 15x)
	\qquad P_6(x) = \frac{1}{16}(231x^6 - 315x^4 + 105x^2 - 5)
$$
Mentre le funzioni associati di Legendre (quelle che entrano nella armoniche sferiche) sono:
$$
P_{l}^m (x) = (1-x^2)^{\frac{m}{2}}\frac{d^m}{dx^m} P_l (x)
$$
\subsection{Armoniche sferiche}

$$
 Y^m_l (\theta,\varphi)= {(-1)^{\frac{m+|m|}{2}}} \sqrt{ \frac{2l+1}{4\pi}
\frac{(l-\mid m \mid)!}{(l+\mid m \mid)!} }
P^{ m}_l(\cos\theta) e^{im\varphi} 
$$
\end{document}
